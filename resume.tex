\documentclass[a4paper]{resume}

\begin{document}

%------------------------------------------------
% TITLE SECTION
%------------------------------------------------
\lastupdated{}
\namesection{Oscar L.}{Downing}{\href{mailto:odowning19@gmail.com}{odowning19@gmail.com} | \mynumber{}
	| \urlstyle{same}\url{www.linkedin.com/in/oscar-downing} | GitHub: \href{https://github.com/qozymandias}{\bf qozymandias}}

%------------------------------------------------
% LEFT COLUMN
%------------------------------------------------
\begin{minipage}[t]{0.33\textwidth} % The left column takes up 33% of the text width of the page
	%------------------------------------------------
	% Education
	%------------------------------------------------
	\section{Education}
	\smallsectionspace{}

	\subsection{University of New South Wales}
	\descript{B.E. (Hons) in Computer Engineering}
	\location{Feb. 2016 --- Aug. 2021 | Sydney, NSW}
	Honours thesis: ``Enhancements to the \href{https://github.com/NICTA/cogent}{\textsc{Cogent}} Property-Based
	Testing framework''. \\ Stephen Robjohns Memorial Rural Engineering Scholar.

	%------------------------------------------------
	% Skills
	%------------------------------------------------
	\section{Skills}
	\smallsectionspace{}

	\subsection{Programming}
	\smallsectionspace{}

	\location{Proficient}
	\rust{} \textbullet{} \clang{} \textbullet{} \cpp{} \\
	\python{} \textbullet{} \bash{} \\
	\typescript{} \textbullet{} \javascript{} \\
	\smallsectionspace{}

	\location{Intermediate}
	\java{} \textbullet{} \kotlin{} \\
	\haskell{} \\
	\vhdl{} \textbullet{} \verilog{} \\
	\LaTeX{} \textbullet{} \matlab{} \\
	\isabelle{} (proof assistant) \\
	\sectionspace{}

	\subsection{Web \& Cloud}
	\smallsectionspace{}
	\xml{} \textbullet{} \json{} \\
	\css{} \textbullet{} \react{} \\
	\aws{} \boto{} \textbullet{} \ectwo{} \textbullet{} \rds{} \\
	\jnciaJunos{} \textbullet{} \jnciaDevOps{} \\
	\sectionspace{}

	\subsection{Tools \& Frameworks}
	\smallsectionspace{}
	\linux{} \textbullet{} \git{} \textbullet{} \docker{} \\
	\cmake{} \textbullet{} \conan{} \textbullet{} \gdb{} \\
	\pytest{} \textbullet{} \asyncio{} \\
	\nodejs{} \textbullet{} \spring{} \\
	\googletest{} \textbullet{} \junit{} \\
	\cucumber{} \\
	\ansible{} \textbullet{} \jenkins{} \\

	%------------------------------------------------
	% Side Projects & Volunteering
	%------------------------------------------------
	\section{Exper. Cont.}
	\smallsectionspace{}

	\runsubsection{Soprano Design}
	\descript{| Graduate Software Engineer}
	\location{Aug. 2020 --- Dec. 2020}
	\vspace{\topsep}
	\textbullet{}
	Contributed to the development and QA of the MEMS platform --- a Communication Platform as a Service. \\
	\textbullet{}
	\java{} web application (Apache Struts), tested with \junit{} and \cucumber{} Test framework. \\
	\smallsectionspace{}

\end{minipage} % The end of the left column
\hfill
% do not delete this comment
%
%----------------------------------------------------------------------------------------
%	RIGHT COLUMN
%----------------------------------------------------------------------------------------
%
\begin{minipage}[t]{0.66\textwidth} % The right column takes up 66% of the text width of the page
	%------------------------------------------------
	% Experience
	%------------------------------------------------
	\section{Experience}
	\smallsectionspace{}

	\runsubsection{Dolby.io}
	\descript{| Software Engineer}
	\location{Dec. 2021 --- Present | Sydney, NSW}
	\vspace{\topsep} % hacky fix only use here
	\begin{tightitemize}
		\item {
		            Contributed to the Dolby Voice Conferencing Server (DVCS) project, a highly performant multi-threaded
		            audio mixing server written in \cpp{} (\cppstd{}/\conan{}/\cmake{}) for \voip{} communications; used
		            by the \dolbyio{} platform, designed for conferencing and virtual worlds communication at high-scale
		            --- thousands of real clients connected and talking. Designed/developed/tested critical new features
		            (Dist Attens).
		      }
		\item {
		            Developed supporting \python{}/\bash{} scripts for testing/CI/CD; including unit/system tests
		            (\googletest{}) and debugged complex multi-threaded problems in \cpp{} with \gdb{}. Improved \oats{}
		            end-to-end testing framework (\python{}/\asyncio{}) which generated \rtp{} and impaired traffic for
		            quality analysis.
		      }
		\item {
		            Improved CI scripts (as DVCS shifted its CI to \gitlab) and \aws{} scripts (\python/\boto) which
		            saved thousands of dollars in \ectwo{} computing costs. Additionally, built a \nodejs{}
		            \typescript{} application for viewing CI/CD pipelines which interfaced with \gitlab{} \rest{} APIs.
		      }
		\item {
		            Improved documentation for onboarding new engineers. Mentored interns, actively learned new technologies,
		            and presented talks internally.
		      }
		\item {
		            Wrote \vagrant{} VM scripts for developers (OS env setup with \bash{} scripting) and \docker{} scripts/images
		            for testing/production.
		      }
		\item {
		            Upgraded build/production environment to \debian{}, rewriting toolchains (\conan{}/\cmake{}) and fixing bugs
		            uncovered by the new compiler; including rewriting packaging scripts and supporting \python scripts.
		      }
		\item {
		            Built client-side and server-side applications (\nodejs{} \typescript{}/ \javascript{}) utilising
		            customer-facing APIs for internal competitions.
		      }
	\end{tightitemize}
	\smallsectionspace{}

	\runsubsection{Atlassian}
	\descript{| Site Reliability Engineer Intern}
	\location{Nov. 2020 --- Feb. 2021 | Sydney, NSW}
	\begin{tightitemize}
		\item {
		            Worked in the Shard Capacity Management team on the Tenant Placement Service project: an internal \aws{}
		            automation microservice, being developed to manage the \rds{} fleet (which hosts data from \jira{} and
		            \conf{} cloud products).
		      }
		\item {
		            Designed and developed software modules for automating manual tasks relating to \rds{} life-cycle
		            management. Specifically: creating a new \rds{} instance, deleting an \rds{} instance, and auto-scaling
		            an \rds{} with low storage. Additionally, tracking and reporting functionality was built to allow for
		            internal audits of \rds{} events. Implemented in \kotlin{}/\spring{}, with a \postgres{} DB (using
		            \jpa{}) and \rest{} APIs; Used in production saving the team from doing these tasks manually and
		            affecting $\num{150000} +$ Atlassian customers.
		      }-
	\end{tightitemize}
	\smallsectionspace{}

\end{minipage}
\end{document}
